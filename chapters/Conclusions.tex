\chapter{Conclusions} \label{chap:Conclusions}
This work was attempting to improve the run time of the homomorphic encryption library, HElib. By applying distribute system techniques, which would suit the intended deployment environment for HElib, we tried to improve the run times of operations being performed by HElib. Two libraries were designed: GPUHElib and DistributedHElib. 

GPUHElib attempted to add GPU functionality to HElib, in order to cut down on the run times for the operations. The design of these operations on the GPU required memory mapping from the CPU to the GPU, overflow considerations for GPU operations, and pipeline techniques be applied. Unfortunately as the tests showed, this design did not perform better than the serial version. This was because the memory transfer times from CPU to GPU were much too large to facilitate a speedup, even though the operation times were much lower. With further work however, this design might become viable.

DistributedHElib applied distributed computing techniques in an attempt to speed up the run time of HElib. This design utilized a master-slave cluster architecture and non-block send and receive function to facilitate concurrent computation. Again as the tests showed, this design failed to perform better than the serial version. This was because the network speed caused a bottleneck, and made the run times drastically slower than the serial version. Similarly, with future work, this design might become a viable option.

While both of these variants were unsuccessful, they have promise, and in the future might be useful in the design of faster homomorphic encryption libraries.