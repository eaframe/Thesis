Homomorphic Encryption, an encryption scheme only developed in the last five years, allows for arbitrary operations to be performed on encrypted data. Using this scheme, a user can encrypt data, and send it to an online service. The online service can then perform an operation on the data and generate an encrypted result. This encrypted result is then sent back to the user, who decrypts it. This decryption produces the same data as if the operation performed by the online service had been performed on the unencrypted data. This is revolutionary because it allows for users to rely on online services, even untrusted online services, to perform operations on their data, without the online service gaining any knowledge from their data.

A prominent implementation of homomorphic encryption is HElib. While one is able to perform homomorphic encryption with this library, there are problems with it. It, like all other homomorphic encryption libraries, is slow relative to other encryption systems. Thus there is a need to speed it up. Because homomorphic encryption will be deployed on online services, many of them distributed systems, it is natural to modify HElib to utilize some of the tools that are available on them in an attempt to speed up run times. Thus two modified libraries were designed: GPUHElib, which utilizes a GPU, and DistributedHElib, which utilizes a distributed computing design. These designs were then tested against the original library to see if they provided any speed up.