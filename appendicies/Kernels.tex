\chapter{Kernels} \label{chap:Kernels}
\section{Addition} \label{sec:KernelAddition}
\subsection{Addition of two DoubleCRT objects}
\begin{lstlisting}[language=C,caption={Addition Kernel for Two DoubleCRT}]
__global__ void vectorAddMod(long *vector_A, 
                            long *vector_B, 
                            long width, 
                            long sub_width, 
                            long offset) {
	long tid;
	tid = offset + blockIdx.x * blockDim.x + threadIdx.x;
	while ((tid-offset) < width) {
	    // extract modulus from array
		uint64_t modulus = 
		        (uint64_t)vector_moduli[tid / sub_width];
		// compute sum
		uint64_t sum = 
		        (uint64_t)vector_A[tid] + 
		        (uint64_t)vector_B[tid];
		// perform modulus operation, store result
		vector_A[tid] = 
		    (long)(sum - modulus * (sum / modulus));
		tid += blockDim.x * gridDim.x;
	}
	return;
}
\end{lstlisting}

\subsection{Addition of a DoubleCRT object and a constant}
\begin{lstlisting}[language=C,caption={Addition Kernel for a DoubleCRT and a constant}]
__global__ void vectorAddMod(long *vector_A, 
                            long width, 
                            long sub_width, 
                            long offset) {
	long tid;
	tid = offset + blockIdx.x * blockDim.x + threadIdx.x;
	while ((tid-offset) < width) {
	    // extract modulus from array
		uint64_t modulus = 
		    (uint64_t)vector_moduli[tid / sub_width];
		// compute sum
		uint64_t sum = 
		    (uint64_t)vector_A[tid] + 
		    (uint64_t)vector_ns[tid / sub_width];
		// perform modulus operation, store result
		vector_A[tid] = 
		    (long)(sum - modulus * (sum / modulus));
		tid += blockDim.x * gridDim.x;
	}
	return;
}
\end{lstlisting}

\section{Subtraction} \label{sec:KernelSubtraction}
\subsection{Subtraction of two DoubleCRT objects}
\begin{lstlisting}[language=C,caption={Subtraction Kernel for Two DoubleCRT}]
__global__ void vectorSubMod(long *vector_A, 
                            long *vector_B, 
                            long width, 
                            long sub_width, 
                            long offset) {
	long tid;
	tid = offset + blockIdx.x * blockDim.x + threadIdx.x;
	while ((tid-offset) < width) {
	    // extract modulus from array
		uint64_t modulus = 
		    (uint64_t)vector_moduli[tid / sub_width];
		// compute difference
		// add modulus to ensure result is > 0, 
	    // will be removed when mod by modulus
		uint64_t diff = 
		    (uint64_t)vector_A[tid] + 
		    modulus - 
		    (uint64_t)vector_B[tid];
		
		// perform modulus operation, store result
		vector_A[tid] = 
		    (long)(diff - modulus * (diff / modulus));
		tid += blockDim.x * gridDim.x;
	}
	return;
}
\end{lstlisting}

\subsection{Subtraction of a DoubleCRT object and a constant}
\begin{lstlisting} [language=C,caption={Subtraction Kernel for a DoubleCRT and a constant}]
__global__ void vectorSubMod(long *vector_A, 
                            long width, 
                            long sub_width, 
                            long offset) {
	long tid;
	tid = offset + blockIdx.x * blockDim.x + threadIdx.x;
	while ((tid-offset) < width) {
	    // extract modulus from array
		uint64_t modulus = 
		    (uint64_t)vector_moduli[tid / sub_width];
		// compuate difference
		// add modulus to ensure result is > 0, 
		// will be removed when mod by modulus
		uint64_t diff = 
		    (uint64_t)vector_A[tid] + 
		    modulus - 
		    (uint64_t)vector_ns[tid / sub_width];
		// perform modulus operation, store result
		vector_A[tid] = 
		    (long)(diff - modulus * (diff / modulus));
		tid += blockDim.x * gridDim.x;
	}
	return;
}
\end{lstlisting}

\section{Multiplication} \label{sec:KernelMultiplication}
\subsection{Multiplication of two DoubleCRT objects}
\begin{lstlisting}[language=C,caption={Multiplication Kernel for Two DoubleCRT}]
__global__ void vectorMultMod(long *vector_A, 
                            long *vector_B, 
                            long width, 
                            long sub_width, 
                            long offset) {
	long tid;
	tid = offset + blockIdx.x * blockDim.x + threadIdx.x;
	while ((tid-offset) < width) {
	    // extract modulus from array
		uint64_t modulus = 
		    (uint64_t)vector_moduli[tid / sub_width];
		// compute intermediate values
		uint64_t a1 = 
		    (uint64_t)vector_A[tid] / 4294967296;
		uint64_t a2 = 
		    (uint64_t)vector_A[tid] - 4294967296 * a1;
		uint64_t b1 = 
		    (uint64_t)vector_B[tid] / 4294967296;
		uint64_t b2 = 
		    (uint64_t)vector_B[tid] - 4294967296 * b1;
        // compute z0, perform modulus
		uint64_t z0 = a2 * b2;
		z0 = z0 - modulus * (z0 / modulus);
        // compute part of z1
		uint64_t p12 = a1 * b2;
        //compute part of z1
		uint64_t p21 = a2 * b1;
        // compute z1, perform modulus
		uint64_t z1 = p12 + p21;
		z1 = z1 - modulus * (z1 / modulus);
		// compute z2
		uint64_t z2 = a1 * b1;
        // loop to get z2*2^32 and z1*2^32
		int i;
		for(i=0; i<32; i++) {
			z1 = z1 * 2;
			if(z1 >= modulus) {
				z1 = 
				    z1 - modulus * 
				    (z1 / modulus);
			}

			z2 = z2 * 2;
			if(z2 >= modulus) {
				z2 = 
				    z2 - modulus * 
				    (z2 / modulus);
			}
		}
        // loop to get z2*2^32*2^32=z2*2^64
		for(i=i; i<64; i++) {
			z2 = z2 * 2;
			if(z2 >= modulus) {
				z2 = 
				    z2 - modulus *
				    (z2 / modulus);
			}
		}
        // compute final value
		uint64_t z = z0 + z1 + z2;
		// perform modulus operation, store result
		vector_A[tid] = 
		    (long)(z - modulus * (z / modulus));
		tid += blockDim.x * gridDim.x;
	}
	return;
}
\end{lstlisting}